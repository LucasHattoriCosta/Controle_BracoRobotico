\documentclass[10pt]{article}

% Use of Lucida Bright 
% Uncomment if Computer Modern
%\usepackage[lucidasmallscale]{lucimatx}
\usepackage[utf8]{inputenc}
\usepackage[brazilian]{babel}
\usepackage[T1]{fontenc}

\usepackage{tabularx,vmargin}
\setpapersize{A4} % Vmargin commands
\setmarginsrb{10mm}{10mm}{10mm}{10mm}{0pt}{10mm}{0pt}{10mm}
%\restylefloat{figure}  % comandos de float.sty para permitir parametro H

\newcommand{\single}{\footnotesize\renewcommand{\baselinestretch}{0.90}\normalsize}
\newcommand{\double}{\footnotesize\renewcommand{\baselinestretch}{1.0}\normalsize}

\usepackage{graphicx}
\usepackage{amsmath}
\usepackage{float}

%\renewcommand{\labelenumi}{{\sf\bfseries [Q. \theenumi]}}

% comment unused packages below
%\usepackage{color}
%\pdfoutput=0 % uncomment this to run PDFTeX in TeX mode
\usepackage{ifpdf}
%\ifx\pdftexversion\undefined %if using TeX
%  \usepackage{graphicx}
%\else %if using PDFTeX
%  \usepackage[pdftex]{graphicx}
%\fi
%\ifpdf %if using PDFTeX in PDF mode
%  \DeclareGraphicsExtensions{.pdf,.png,.mps}
  \usepackage{pgf}
  \usepackage{tikz}
%\else %if using TeX or PDFTeX in TeX mode
  \usepackage{graphicx}
  \DeclareGraphicsExtensions{.eps,.bmp}
  \DeclareGraphicsRule{.emf}{bmp}{}{}% declare EMF filename extension
  \DeclareGraphicsRule{.png}{bmp}{}{}% declare PNG filename extension
  \usepackage{pgf}
  \usepackage{tikz}
  \usepackage{pstricks}%variant: \usepackage{pst-all}
%\fi

%\usepackage{epic,bez123}
\usepackage{float}

\usepackage{array}
\newcolumntype{M}[1]{>{\centering\arraybackslash}m{#1}}
\newcolumntype{N}{@{}m{0pt}@{}}

\usepackage{wrapfig}% package for wrapfigure environment

\newcommand{\espacomeio}{\vspace*{0.5cm}}
\newcommand{\espacoum}{\vspace*{1.0cm}}
\newcommand{\espacodois}{\vspace*{2.0cm}}
\newcommand{\espacotres}{\vspace*{3.0cm}}
\newcommand{\espacoquatro}{\vspace*{4.0cm}}
\newcommand{\espacocinco}{\vspace*{5.0cm}}

\usepackage[UKenglish]{isodate}
%\usepackage{draft}

\usepackage{lastpage}
\usepackage{fancyhdr}
\pagestyle{fancy}
\fancyhead[L]{}
%\fancyhead[C]{\framebox{DRAFT - \today}}
\fancyhead[C]{}
\fancyhead[R]{}
\fancyfoot[L]{}
\fancyfoot[C]{\thepage\ of \pageref{LastPage}}
\fancyfoot[R]{VISTO:\rule{2cm}{0.5pt}}
\renewcommand{\headrulewidth}{0.0pt}
\renewcommand{\footrulewidth}{0.0pt}

%%%%%%%%%%%%%% TIKZ

\usepackage{tikz}
\usetikzlibrary{calc,patterns,decorations.pathmorphing,arrows.meta,decorations.markings,shapes,fit,arrows}
% Definition of blocks:
\tikzstyle{block} = [draw, fill=white, rectangle, 
minimum height=4em, minimum width=3em]
\tikzstyle{sum} = [draw, fill=white, circle, node distance=2cm]
\tikzstyle{input} = [coordinate]
\tikzstyle{output} = [coordinate]
\tikzstyle{pinstyle} = [pin edge={to-,thin,black}]

\usepackage{amsthm}
\theoremstyle{plain}
%\usepackage{thmbox}
\usepackage[pdftex,color,outerbars]{changebar}
\newtheorem{exercicio}{Para você fazer}
%\newenvironment{cexercicio}[1][]{\begin{changebar}\if\relax\detokenize{#1}\relax\begin{exercicio}\else
%			\begin{exercicio}[#1]\fi}
%			{\end{exercicio}\end{changebar}}
\newenvironment{cexercicio}[1][]{\if\relax\detokenize{#1}\relax\begin{exercicio}\else
			\begin{exercicio}[#1]\fi}
			{\end{exercicio}}
		
\usepackage{afterpage}

\begin{document}

\vspace*{-1cm}

\begin{center}
	{\bfseries\Large PMR3404 Controle I - Experiência 1:
		             Matlab, Python Control System Toolbox, Controladores PID}
\end{center}	

\begin{center}
\begin{tabular}{| p{9cm}  p{2.5cm}  p{1.7cm}| M{2.5cm} |} \hline &&& \\
	& & & {\bf NOTA} \\ [0pt]
	{\bf Alunos do grupo} & & & \\ [5pt]
	1. Nome: \hrulefill & NUSP: \hrulefill & Turma: \hrulefill &  \\ [10pt]
	2. Nome: \hrulefill & NUSP: \hrulefill & Turma: \hrulefill &  \\ [10pt]
	3. Nome: \hrulefill & NUSP: \hrulefill & Turma: \hrulefill &  \\ [10pt] 
	Data: \rule{4mm}{0.5pt} /\rule{4mm}{0.5pt} /\rule{4mm}{0.5pt} \hspace*{20pt} Reposição? (\;\;\;) S (\;\;\;) N & & & \\ [10pt] \hline
\end{tabular}
\end{center}

\single
\setlength{\parskip}{2pt}

\section{Entregas do relatório}

Esse texto é uma combinação de apostila, roteiro experimental e espaços
do tipo formulário para desenvolver as respostas e também inserir as figuras.
Você deve preencher as seções que se encontram distribuídas pelo texto e são identificadas com o texto:
{\bf Para você fazer}.


O relatório deve {\bf OBRIGATORIAMENTE} ser feito nesse mesmo formulário. As seguintes alternativas podem
ser utilizadas:
\begin{enumerate}
\item O relatório pode ser preenchido a mão. Os gráficos podem ser impressos em papel e anexados
      ao final do relatório devidamente identificados.
      Ao finalizar todas as atividades você deve escanear o relatório (Utilize um aplicativo de celular como CamScanner, Adobe Scan, etc.) e salvar num arquivo PDF.
\item Alternativamente você pode preencher o formulário utilizando um programa para edição de PDFs.
\item Para os usuário de Latex/Overleaf a fonte *.tex tambem é fornecida porém com a utilização da fonte
      Computer Modern.
\end{enumerate}

\vspace*{12pt}
{\bf ATENÇÃO} 

{\bf O relatório pode ser realizado invidualmente ou em duplas.}

{\bf Relatórios em papel não serão aceitos. }

{\bf Somente arquivos PDF serão aceitos.}

{\bf O relatório completo deve ser composto de um único arquivo PDF.}

\section{Roteiro}

Nessa primeira aula prática vamos introduzir a utilização do software MATLAB (Ou alternativamente Python) para o projeto de sistemas
de controle em malha fechada utilizando controladores PID.
As técnicas de projeto de sistemas de controle utilizando os requisitos de desempenho será o tema da segunda aula.

Os seguintes temas serão abordados:
\begin{itemize}
	\item Modelagem de um sistema de controle de posicionamento de antena parabólica,
	\item Obtenção das funções de transferência de velocidade e de posição do sistema,
	\item Experimentos com controladores P, PI, PD serão realizados visando principalmente
	      observar o efeito de cada componente: P, I e D sobre cada sistema.
\end{itemize}

\section{Introdução}

\subsection{A Filosofia de projeto de sistemas de controle}

\subsection{Definição do sistema de controle em estudo}

\section{Funções de transferência}

\subsection{Sistema de controle de velocidade}

Partindo do diagrama de blocos do sistema de controle de velocidade (Figura \ref{fig:blocovelang}) é possível atingir a topologia do sistema de controle em malha fechada standard
(Figura \ref{fig:standardcl}) através da manipulação do diagrama de blocos.

Por exemplo, se admitirmos que $K_{tac}=K_{pot2}$ podemos substituir os dois blocos por um único
bloco colocado após o somador.
É necessário agora estabelecer dois blocos que denominaremos controlador $H_\omega(s)$ e planta $G_\omega(s)$.

O sistema de controle ficará definido como ilustrado na Figura \ref{fig:rearranjovel}

\begin{figure}[H]\centering
\begin{tikzpicture}[auto, node distance=2.2cm,>=latex']
\node [input, name=input] {};
%\node [block, right of=input, label={[above,align=center]{{\small Potenciômetro}}}] (ktac1) {$K_{pot2}$};
\node [sum, right of=input] (sum) {$+$};

\node [block, right of=sum, label={[above,align=center]{{\small Tacômetro}}}] (ktac1) {$K_{tac}$};

\node [block, right of=ktac1, label={[above,align=center]{{\small Pré-Amp}}}] (controller) {$K$};
\node [block, right of=controller, label={[above,align=center]{{\small Amp Pot}}}] (poweramp) {$\frac{K_1}{s+a}$};
\node [block, right of=poweramp, label={[above,align=center]{{\small Motor,Carga}}}] (system) {$\frac{K_m}{s+a_m}$};
\node [block, right of=system,label={[above,align=center]{{\small Engrenagens}}}] (engrenagens) {$K_g$};
\node [input, below of=poweramp] (ktac2) {};
\node [output, right of=engrenagens] (output) {};
\node[draw,dashed,color=gray!70, inner sep=15pt,thick,fit=(poweramp) (system) (engrenagens), label={[above,align=center]{{$G_\omega(s)$}}}] (F1){};
\node[draw,color=gray!70, dashed,inner sep=15pt,thick,fit=(ktac1) (controller), label={[above,align=center]{{$H_\omega(s)$}}}] (F2){};
%   Conexao de blocos
\draw [draw,->] (input) -- node {$\Omega_i(s)$} (sum);
%\draw [->] (ktac) -- node {$V_i(s)$} (sum);
\draw [->] (sum) -- node {$\Omega_e(s)$} (ktac1);
\draw [->] (ktac1) -- node {$V_e(s)$} (controller);
\draw [->] (controller) -- node {$V_p(s)$} (poweramp);
\draw [->] (poweramp) -- node {$E_a(s)$} (system);
\draw [->] (system) -- node {$\Omega_m(s)$} (engrenagens);
\draw [->] (engrenagens) -- node [name=omegao] {$\Omega_o(s)$} (output);

%\draw [->] (omegao) |- node [above,pos=0.79] {} (ktac2) ;
%\draw [->] (ktac2) -| node [pos=0.95] {$-$} node [near end] {} (sum);

\draw [-] (omegao) |- node [above,pos=0.79] {} (ktac2) ;
\draw [->] (ktac2) -| node [pos=0.95] {$-$} node [near end] {} (sum);
\end{tikzpicture}
\caption{Redução do diagrama de blocos do sistema de controle de velocidade.}
\label{fig:rearranjovel}
\end{figure}

O controlador será definido através da seguinte forma geral:
\begin{equation}
H_\omega(s)=K_{tac}H_1(s).
\end{equation}
Para o nosso propósito $H_1(s)$ pode ser uma constante $K$ como ilustrado na Figura \ref{fig:rearranjovel} mas estamos interessados em controladores mais
complexos do tipo PID.

A planta pode ser escrita como:
\begin{equation}
\frac{\Omega_o(s)}{V_p(s)}=G_\omega(s)=\frac{K_1 K_m K_g}{(s+a)(s+a_m)}
\end{equation}


%\framebox{{\Large\bf Para você fazer $\blacktriangleright$}}

%{\Large\bf \vspace*{12pt} [ Para você fazer $\blacktriangleright$}


\begin{cexercicio} 
	
\begin{enumerate}
\item Utilizar o script {\tt FTVelocidadeAntena.m} ou {\tt FTVelocidadeAntena.py} 
\item Calcule a função de transferência do sistema com valores numéricos:
\begin{center}
	\begingroup \large
	\renewcommand{\arraystretch}{4} % Default value: 1
	\begin{tabular}{|c|p{10cm}|}\hline
		\( \displaystyle G_\omega(s)=\frac{\Omega(s)}{V_p(s)} = \frac{K_c}{s^2+c_1s+c_2} \)  &   \\ \hline		  		  
	\end{tabular}
	\endgroup
\end{center}

\item Reescreva a função de transferência evidenciando os pólos:
\begin{center}
	\begingroup \large
	%\setlength{\tabcolsep}{100pt} % Default value: 6pt
	\renewcommand{\arraystretch}{4} % Default value: 1
	\begin{tabular}{|c|p{10cm}|}\hline
		\( \displaystyle G_\omega(s)=\frac{\Omega(s)}{V_p(s)}= \frac{K_d}{(s+d_1)(s+d_2)} \)  &   \\ \hline		  		  
	\end{tabular}
	\endgroup
\end{center}

\item Reescreva a função de transferência evidenciando as constantes de tempo:
\begin{center}
	\begingroup \large
	%\setlength{\tabcolsep}{100pt} % Default value: 6pt
	\renewcommand{\arraystretch}{4} % Default value: 1
	\begin{tabular}{|c|p{10cm}|}\hline
		\( \displaystyle G_\omega(s)=\frac{\Omega(s)}{V_p(s)}= \frac{K_T}{(T_1s+1)(T_2s+1)} \)  &   \\ \hline		  		  
	\end{tabular}
	\endgroup
\end{center}

\item Reescreva a função de transferência através da expansão em frações parciais:
\begin{center}
	\begingroup \large
	%\setlength{\tabcolsep}{100pt} % Default value: 6pt
	\renewcommand{\arraystretch}{4} % Default value: 1
	\begin{tabular}{|c|p{10cm}|}\hline
		\( \displaystyle G_\omega(s)=\frac{\Omega(s)}{V_p(s)}= \frac{A_1}{(s+d_1)} + \frac{A_2}{(s+d_2)} \)  &   \\ \hline		  		  
	\end{tabular}
	\endgroup
\end{center}

\item Identifique os pólos do sistema $G_\omega(s)$:
	\begin{center}
	\begingroup \large
	%\setlength{\tabcolsep}{36pt} % Default value: 6pt
	\renewcommand{\arraystretch}{2} % Default value: 1
	\begin{tabular}{|c|p{5cm}|}\hline
		pólo 1  &   \\ \hline
		pólo 2  &   \\ \hline	  		  
	\end{tabular}
	\endgroup
\end{center}

\item Esboce abaixo a resposta a degrau do sistema:
    \begin{center}
	\begin{tikzpicture}
	\draw [thick,step=0.5, gray!20] (0,0) grid (15,6);
	\end{tikzpicture}
    \end{center}
\end{enumerate}
\end{cexercicio}

\subsection{Sistema de controle de posição angular}

Da mesma forma, é possível partir do diagrama de blocos do sistema de controle de posição (Figura \ref{fig:bloco}) e atingir a topologia do sistema de controle em malha fechada standard
(Figura \ref{fig:standardcl}) através da manipulação do diagrama de blocos.


É necessário agora estabelecer dois blocos que denominaremos controlador $H_\theta(s)$ e planta $G_\theta(s)$.

O sistema de controle ficará definido como ilustrado na Figura \ref{fig:rearranjopos}:

\begin{figure}[H]\centering
\input{diagblocoposicaoRearranjoDashed.tex}
	\caption{Redução do diagrama de blocos do sistema de controle de posição angular.}
	\label{fig:rearranjopos}
\end{figure}

O controlador será definido através da seguinte forma geral:
\begin{equation}
H_\theta(s)=K_{pot}H_1(s).
\end{equation}
Para o nosso propósito $H_1(s)$ será representado por controladores c PID.

A planta pode ser escrita como:
\begin{equation}
\frac{\Theta_o(s)}{V_p(s)}=G_\theta(s)=\frac{K_1 K_m K_g}{s(s+a)(s+a_m)}
\end{equation}


%\framebox{{\Large\bf Para você fazer $\blacktriangleright$}}

%{\Large\bf \vspace*{12pt} [ Para você fazer $\blacktriangleright$}

\begin{cexercicio}{}
\begin{enumerate}
	\item Utilizar o script {\tt FTPosicaoAntena.m} ou {\tt FTPosicaoAntena.py} 
	\item Calcule a função de transferência do sistema com valores numéricos:
	\begin{center}
		\begingroup \large
		\renewcommand{\arraystretch}{4} % Default value: 1
		\begin{tabular}{|c|p{10cm}|}\hline
			\( \displaystyle G_\theta(s)=\frac{\Theta(s)}{V_p(s)} = \frac{K_c}{s^3+c_1s^2+c_2s+c_3} \)  &   \\ \hline		  		  
		\end{tabular}
		\endgroup
	\end{center}
	
	\item Reescreva a função de transferência evidenciando os pólos:
	\begin{center}
		\begingroup \large
		%\setlength{\tabcolsep}{100pt} % Default value: 6pt
		\renewcommand{\arraystretch}{4} % Default value: 1
		\begin{tabular}{|c|p{10cm}|}\hline
			\( \displaystyle G_\theta(s)=\frac{\Theta(s)}{V_p(s)}= \frac{K_d}{s(s+d_1)(s+d_2)} \)  &   \\ \hline		  		  
		\end{tabular}
		\endgroup
	\end{center}
		
	\item Identifique os pólos do sistema $G_\theta(s)$:
	\begin{center}
		\begingroup \large
		%\setlength{\tabcolsep}{36pt} % Default value: 6pt
		\renewcommand{\arraystretch}{2} % Default value: 1
		\begin{tabular}{|c|p{5cm}|}\hline
			pólo 1  &   \\ \hline
			pólo 2  &   \\ \hline
			pólo 3 & \\ \hline	  		  
		\end{tabular}
		\endgroup
	\end{center}
	
	\item Esboce abaixo a resposta a degrau do sistema:
	\begin{center}
		\begin{tikzpicture}
		\draw [thick,step=0.5, gray!20] (0,0) grid (15,6);
		\end{tikzpicture}
	\end{center}
\end{enumerate}
\end{cexercicio}

\section{Controladores PID}

O controlador PID pode ser escrito de diversas formas. Aqui vamos adotar o seguinte formato:
\begin{equation}
H_1(s) = K_p\left( 1 + \frac{1}{T_is}+T_d\frac{s}{\frac{T_d}{N}s+1} \right),
\end{equation}
onde $K_p$ é denominada constante proporcional, $T_i$ é o tempo integral, $T_d$ é o tempo derivativo e
$N$ é o parâmetro do filtro de 1a. ordem que possui um pólo em $s=-N/T_d$.
O filtro é necessário para garantir estabilidade da parte derivativa.

O seu uso prático não necessariamente utiliza os três termos simultaneamente, dessa forma  podemos
observar a utilização de variações como controladores P, PI, PD e PID.

O sucesso industrial dos controladores PID se explica pelo fato de possuir flexibilidade suficiente para
obter uma solução de controle para quase qualquer processo industrial.

Para demonstrar o papel de cada parâmetro do controlador PID realizaremos experimentos de simulação
dos sistemas de controle de velocidade e de posição.

Utilizaremos para todos os experimentos de simulação os mesmos {\tt scripts}:
\begin{enumerate}
	\item {\tt ControleDeVelocidade\_PID.m},\;{\tt Exp1\_ControleDeVelocidade\_PID.py},\; {\tt Exp1\_ControleDeVelocidade\_PID.ipynb},
	\item {\tt ControleDePosicao\_PID.m},\;{\tt Exp1\_ControleDePosicao\_PID.py},\; {\tt Exp1\_ControleDePosicao\_PID.ipynb}.
\end{enumerate}
{\bf Os parâmetros dos controladores PID devem ser alterados adequadamente para cada situação.}

\section{Controlador Proporcional}

O controlador proporcional é o controlador mais simples possível e é definido aqui como
$H_1(s)=K_p$.

Algumas características dessa ação de controle:
\begin{itemize}
\item O ganho proporcional $K_p$ multiplica o sinal de controle proporcionalmente ao erro,
\item Com o aumento do ganho proporcional o sistema reage mais rápido porém pode eventualmente
      levar  a um sistema mais oscilatório, com maior sobressinal,
\item Eventualmente o aumento do ganho proporcional leva o sistema para uma região instável.
\item Ganhos proporcionais elevados reduzem a faixa linear de operação dos atuadores devido ao
      fenômeno de saturação.
\end{itemize}

A seguir algumas características desse controlador poderão ser observadas através de
simulações do sistema de controle de velocidade e de posição

\newpage

\subsection{Controle de velocidade}

\begin{cexercicio}{}
\begin{enumerate}
	\item Considerando que $H_\omega(s)=K_{tac}K_p$ e que:
	      \begin{equation}
	      G_{\omega}(s) = \frac{K_1 K_m K_g}{(s+a)(s+a_m)},
	      \end{equation}
	      calcule o erro estático $\omega_{e_{ss}}$ do sistema de controle em malha fechada (Veja \ref{fig:rearranjovel}) para uma entrada
	      degrau $\Omega_i(s)=A/s$.
	      Utilize os seguintes resultados (As variáveis aqui se referem à Figura \ref{fig:standardcl}):
	      \begin{itemize}
	      	\item \begin{equation} e_{ss} = \lim_{t\rightarrow\infty}e(t) = \lim_{s\rightarrow 0} s E(s)\end{equation},
	      	\item \begin{equation}\frac{E(s)}{R(s)}=\frac{1}{1+GH(s)}.\end{equation}
	      \end{itemize}
      
          \espacoquatro

    \item Utilizando o script de controle de velocidade realize simulações do sistema de 
          controle em malha fechada com três valores do ganho proporcional do controlador
          $K_p=\{ 3.0, 5.0, 7.0 \}$.
          
    \item Identifique os pólos da malha aberta $GH_\omega(s)$ para cada um dos valores de $K_p$:
      \begin{center}
      	\begingroup \large
      	\setlength{\tabcolsep}{30pt} % Default value: 6pt
      	\renewcommand{\arraystretch}{2} % Default value: 1
      	\begin{tabular}{|c|c|c|c|}\hline
           $K_p$  & $K_{p_1}$ & $K_{p_2}$ & $K_{p_3}$   \\ \hline
      		pólo 1  &        &        &        \\ \hline
      		pólo 2  &        &        &        \\ \hline	  		  
      	\end{tabular}
      	\endgroup
      \end{center}
      
    \item Identifique os pólos da malha fechada para cada um dos valores de $K_p$:
\begin{center}
	\begingroup \large
	\setlength{\tabcolsep}{30pt} % Default value: 6pt
	\renewcommand{\arraystretch}{2} % Default value: 1
	\begin{tabular}{|c|c|c|c|}\hline
		$K_p$  & $K_{p_1}$ & $K_{p_2}$ & $K_{p_3}$   \\ \hline 
		pólo 1  &        &        &        \\ \hline
		pólo 2  &        &        &        \\ \hline	  		  
	\end{tabular}
	\endgroup
\end{center}

	    \item Compare a posição dos pólos de malha fechada em relação à posição
              dos pólos de malha aberta.

        \espacoquatro

    \item Esboce abaixo a resposta transitória para uma entrada do tipo degrau unitário.
     \begin{center}
     	\begin{tikzpicture}
     	\draw [thick,step=0.5, gray!20] (0,0) grid (15,6);
     	\end{tikzpicture}
     \end{center}
 
    \item Preencha na tabela abaixo o  erro estático para cada valor de $K_p$:

\begin{center}
	\begingroup \large
	\setlength{\tabcolsep}{30pt} % Default value: 6pt
	\renewcommand{\arraystretch}{2} % Default value: 1
	\begin{tabular}{|c|c|}\hline
		$K_p$  & $\omega_{e_{ss}}$ \\ \hline
		 $K_{p_1}$ &       \\ \hline
		 $K_{p_2}$ &       \\ \hline	  
		 $K_{p_3}$ &       \\ \hline	  		  
	\end{tabular}		  
	\endgroup
\end{center}

\item Anote na tabela abaixo os valores obtidos para o tempo de subida $t_r$, tempo de acomodação
      $t_s$, Máximo sobresinal $M_p$ da variável $\omega_o(t)$ e $\max v_p(t)$ (Valor máximo da saída do controlador $v_p(t)$)
\begin{center}
	\begingroup \large
	\setlength{\tabcolsep}{36pt} % Default value: 6pt
	\renewcommand{\arraystretch}{2} % Default value: 1
	\begin{tabular}{|c|c|c|c|c|}\hline
		$K_p$ & $t_r$ &  $t_s$ & $M_p$ & $\max v_p(t)$ \\ \hline
		$K_{p_1}$  & & & & \\ \hline
		$K_{p_2}$  & & & & \\ \hline
		$K_{p_3}$  & & & & \\ \hline		  		  
	\end{tabular}
	\endgroup
\end{center}

\item Como as grandezas $t_r$, $t_s$, $M_p$ e $\max v_p(t)$ variam com a variação de $K_p$ ?

      \espacoquatro

\end{enumerate}
\end{cexercicio}

\subsection{Controle de posição}
O sistema de controle de posição possui na malha aberta $GH_\theta(s)$  um integrador $1/s$ adicional
e uma constante do potenciômetro $K_{pot}$ da mesma ordem de magnitude de $K_{tac}$.

Será verificado que a introdução do integrador torna o sistema sistema mais lento
o que requer a utilização de um ganho $K_p$ bem maior.
Entretanto o integrador transforma o sistema em um sistema do tipo 1.
Como consequência o erro $\theta_{e_{ss}}$ se torna nulo para um sinal de 
referência do tipo degrau.

\begin{cexercicio}{}
	\begin{enumerate}
		\item Considerando que $H_\theta(s)=K_{pot}K_p$ e que:
		\begin{equation}
		G_{\theta}(s) = \frac{K_1 K_m K_g}{s(s+a)(s+a_m)},
		\end{equation}
		calcule o erro estático $\theta_{e_{ss}}$ do sistema de controle em malha fechada (Veja Figura \ref{fig:rearranjopos}) para uma entrada
		degrau $\Theta_i(s)=A/s$.
		
		\espacoquatro
		
		\item Utilizando o script de controle de posição realize simulações do sistema de 
		controle em malha fechada com três valores do ganho proporcional do controlador
		$K_p=\{ 10.0, 20.0, 50.0\}$.
		
		\item Identifique os pólos da malha aberta $GH_\theta(s)$ para cada um dos valores de $K_p$:
		\begin{center}
			\begingroup \large
			\setlength{\tabcolsep}{30pt} % Default value: 6pt
			\renewcommand{\arraystretch}{2} % Default value: 1
			\begin{tabular}{|c|c|c|c|}\hline
				 $K_p$  & $K_{p_1}$ & $K_{p_2}$ & $K_{p_3}$   \\ \hline 
				pólo 1  &        &        &        \\ \hline
				pólo 2  &        &        &        \\ \hline
				pólo 3  &        &        &        \\ \hline					  		  
			\end{tabular}
			\endgroup
		\end{center}
		
		\item Identifique os pólos da malha fechada para cada um dos valores de $K_p$:
		\begin{center}
			\begingroup \large
			\setlength{\tabcolsep}{30pt} % Default value: 6pt
			\renewcommand{\arraystretch}{2} % Default value: 1
			\begin{tabular}{|c|c|c|c|}\hline
				 $K_p$  & $K_{p_1}$ & $K_{p_2}$ & $K_{p_3}$   \\ \hline
				pólo 1  &        &        &        \\ \hline
				pólo 2  &        &        &        \\ \hline
				pólo 3  &        &        &        \\ \hline					  		  
			\end{tabular}
			\endgroup
		\end{center}
		
	    \item Compare a posição dos pólos de malha fechada em relação à posição
                dos pólos de malha aberta.

              \espacoquatro

		\item Esboce abaixo a resposta transitória para uma entrada do tipo degrau unitário.
		\begin{center}
			\begin{tikzpicture}
			\draw [thick,step=0.5, gray!20] (0,0) grid (15,6);
			\end{tikzpicture}
		\end{center}
		
    \item Preencha na tabela abaixo o  erro estático para cada valor de $K_p$:

\begin{center}
	\begingroup \large
	\setlength{\tabcolsep}{30pt} % Default value: 6pt
	\renewcommand{\arraystretch}{2} % Default value: 1
	\begin{tabular}{|c|c|}\hline
		$K_p$  & $\theta_{e_{ss}}$ \\ \hline
		$K_{p_1}$  &       \\ \hline
		$K_{p_2}$  &       \\ \hline	  
		 $K_{p_3}$ &       \\ \hline	  		  
	\end{tabular}		  
	\endgroup
\end{center}	
		
		\item Anote na tabela abaixo os valores obtidos para o tempo de subida $t_r$, tempo de acomodação
		$t_s$, Máximo sobresinal $M_p$ da variável $\omega_o(t)$ e $\max v_p(t)$ (Valor máximo da saída do controlador
		$v_p(t)$)
		\begin{center}
			\begingroup \large
			\setlength{\tabcolsep}{36pt} % Default value: 6pt
			\renewcommand{\arraystretch}{2} % Default value: 1
			\begin{tabular}{|c|c|c|c|c|}\hline
				$K_p$ & $t_r$ &  $t_s$ & $M_p$ & $\max v_p(t)$ \\ \hline
				$K_{p_1}$ & & & & \\ \hline
				$K_{p_2}$ & & & & \\ \hline
				$K_{p_3}$ & & & & \\ \hline		  		  
			\end{tabular}
			\endgroup
		\end{center}
	
	  \item Como as grandezas $t_r$, $t_s$, $M_p$ e $\max v_p(t)$ variam com a variação de $K_p$ ?

           \espacoquatro
			
	\end{enumerate}
\end{cexercicio}

\section{Controlador proporcional integral}

O controlador PI pode ser escrito da seguinte forma:
\begin{equation}
H_1(s) = K_p \left(1+\frac{1}{T_is} \right)
\end{equation}

O controlador PI combina o efeito proporcional e integral no mesmo controlador.

Sabemos pela seção anterior que para o controle de velocidade o integrador é fundamental
para transformar o sistema de tipo 0 para um sistema de tipo 1 e dessa forma
atingir erro estático $\omega_{e_{ss}}=0$ para sinais de referência do tipo degrau.

Para o controle de posição o sistema passa a ser do tipo 2 o que permite atingir erro 
estatico $\omega_{e_{ss}}=0$ para sinais de referência dos tipos degrau e rampa.

A presença de um segundo integrador no sistema de controle de posição torna a escolha
dos parâmeros $K_p$ e $T_i$ bem mais difícil.
Será ilustrado um conjunto de parâmetros que movimenta os pólos para uma região instável.

\subsection{Controle de velocidade} 
\begin{cexercicio}{}
	\begin{enumerate}
		\item Considerando que:
		\begin{equation}
		 H_\omega(s)=K_{tac} K_p \left(1+\frac{1}{T_is}\right)
		 \end{equation}
		 e
		\begin{equation}
		G_{\omega}(s) = \frac{K_1 K_m K_g}{(s+a)(s+a_m)},
		\end{equation}
		calcule o erro estático $\omega_{e_{ss}}$ do sistema de controle em malha fechada para uma entrada
		degrau $\Omega_i(s)=A/s$.
		
		\espacoquatro
		
		\item Utilizando o script de controle de velocidade realize simulações do sistema de 
		controle em malha fechada com três valores do tempo integral do controlador
		$T_i=\{ 0.1, 0.2, 0.5\}$ e valor de $K_p=5.0$.
				
		\item Escreva a função de transferência do sistema em malha aberta com valores numéricos:
	     \begin{center}
		    \begingroup \large
		   	\renewcommand{\arraystretch}{4} % Default value: 1
			\begin{tabular}{|c|p{10cm}|}\hline
				\( \displaystyle GH_\omega(s) \)  &   \\ \hline		  		  
			\end{tabular}
			\endgroup
	     \end{center}
		
		\item Identifique os pólos de malha aberta $GH_\omega(s)$ para cada um dos valores de $T_i$:
		\begin{center}
			\begingroup \large
			\setlength{\tabcolsep}{30pt} % Default value: 6pt
			\renewcommand{\arraystretch}{2} % Default value: 1
			\begin{tabular}{|c|c|c|c|}\hline
				        & $T_{i_1}$ & $T_{i_2}$ & $T_{i_3}$   \\ \hline
				pólo 1  &        &        &        \\ \hline
				pólo 2  &        &        &        \\ \hline
				pólo 3  &        &        &        \\ \hline
%				pólo 4  &        &        &        \\ \hline				
				zero    &        &        &        \\ \hline					  		  
			\end{tabular}
			\endgroup
		\end{center}

		\item Escreva a função de transferência em malha fechada do sistema com valores numéricos:
        \begin{center}
	    \begingroup \large
	    \renewcommand{\arraystretch}{4} % Default value: 1
	    \begin{tabular}{|c|p{10cm}|}\hline
		   \( \displaystyle \frac{GH_\omega(s)}{1+GH_\omega(s)} \)  &   \\ \hline		  		  
	    \end{tabular}
	    \endgroup
        \end{center}
    
		\item Identifique os pólos da malha fechada para cada um dos valores de $K_p$:
		\begin{center}
			\begingroup \large
			\setlength{\tabcolsep}{30pt} % Default value: 6pt
			\renewcommand{\arraystretch}{2} % Default value: 1
			\begin{tabular}{|c|c|c|c|}\hline
				        & $T_{i_1}$ & $T_{i_2}$ & $T_{i_3}$   \\ \hline
				pólo 1  &        &        &        \\ \hline
				pólo 2  &        &        &        \\ \hline
				pólo 3  &        &        &        \\ \hline
%				pólo 4  &        &        &        \\ \hline				
                zero    &        &        &        \\ \hline					  		  
			\end{tabular}
			\endgroup
		\end{center}
		
	    \item Compare a posição dos pólos de malha fechada em relação à posição
              dos pólos de malha aberta.

              \espacoquatro

		\item Esboce abaixo a resposta transitória para uma entrada do tipo degrau unitário.
		\begin{center}
			\begin{tikzpicture}
			\draw [thick,step=0.5, gray!20] (0,0) grid (15,6);
			\end{tikzpicture}
		\end{center}
		
		\item Anote na tabela abaixo os valores obtidos para o tempo de subida $t_r$, tempo de acomodação
		$t_s$, Máximo sobresinal $M_p$ da variável $\omega_o(t)$ e $\max v_p(t)$ (Valor máximo da saída do controlador
		$v_p(t)$)
		\begin{center}
			\begingroup \large
			\setlength{\tabcolsep}{36pt} % Default value: 6pt
			\renewcommand{\arraystretch}{2} % Default value: 1
			\begin{tabular}{|c|c|c|c|c|}\hline
				$T_i$ & $t_r$ &  $t_s$ & $M_p$ & $\max v_p(t)$ \\ \hline
				$T_{i_1}$ & & & & \\ \hline
				$T_{i_2}$ & & & & \\ \hline
				$T_{i_3}$ & & & & \\ \hline		  		  
			\end{tabular}
			\endgroup
		\end{center}
		
	  \item Como as grandezas $t_r$, $t_s$, $M_p$ e $\max v_p(t)$ variam com a variação de $T_i$ ?

      \espacoquatro
	  \end{enumerate}
\end{cexercicio}

\subsection{Controle de posição}

Aqui os controladores PI utilizarão os mesmo parâmetros utilizados para o controle de velocidade.
{\bf A intenção é mostrar que esse sistema se torna facilmente instável}.  

\begin{cexercicio}{}
	\begin{enumerate}
		\item Considerando que:
        \begin{equation}
            H_\theta(s)=K_{pot} K_p \left(1+\frac{1}{T_is}\right)
        \end{equation}
        e
        \begin{equation}
         G_{\theta}(s) = \frac{K_1 K_m K_g}{s(s+a)(s+a_m)},
        \end{equation}		
		calcule o erro estático $\theta_{e_{ss}}$ do sistema de controle em malha fechada para uma entrada
		degrau $\Theta_i(s)=A/s$.
		
		\espacoquatro
		
		\item Utilizando o script de controle de posição realize simulações do sistema de 
		controle em malha fechada com três valores do tempo integral do controlador
		$T_i=\{ 0.1, 0.2, 0.5\}$ e valor de $K_p=5.0$.
		
		\item Escreva a função de transferência do sistema com valores numéricos:
		\begin{center}
			\begingroup \large
			\renewcommand{\arraystretch}{4} % Default value: 1
			\begin{tabular}{|c|p{10cm}|}\hline
				\( \displaystyle GH_\theta(s) \)  &   \\ \hline		  		  
			\end{tabular}
			\endgroup
		\end{center}
		
		\item Identifique os pólos de malha aberta $GH_\theta(s)$ para cada um dos valores de $T_i$:
		\begin{center}
			\begingroup \large
			\setlength{\tabcolsep}{30pt} % Default value: 6pt
			\renewcommand{\arraystretch}{2} % Default value: 1
			\begin{tabular}{|c|c|c|c|}\hline
				$K$  & $T_{i_1}$ & $T_{i_2}$ & $T_{i_3}$   \\ \hline
				pólo 1  &        &        &        \\ \hline
				pólo 2  &        &        &        \\ \hline
				pólo 3  &        &        &        \\ \hline
				pólo 4  &        &        &        \\ \hline
				zero    &        &        &        \\ \hline					  		  
			\end{tabular}
			\endgroup
		\end{center}
		
		\item Escreva a função de transferência em malha fechada do sistema com valores numéricos:
		\begin{center}
			\begingroup \large
			\renewcommand{\arraystretch}{4} % Default value: 1
			\begin{tabular}{|c|p{10cm}|}\hline
				\( \displaystyle \frac{GH_\theta(s)}{1+GH_\theta(s)} \)  &   \\ \hline		  		  
			\end{tabular}
			\endgroup
		\end{center}
		
		\item Identifique os pólos da malha fechada para cada um dos valores de $K_p$:
		\begin{center}
			\begingroup \large
			\setlength{\tabcolsep}{30pt} % Default value: 6pt
			\renewcommand{\arraystretch}{2} % Default value: 1
			\begin{tabular}{|c|c|c|c|}\hline
				$K_p$  & $K_{p_1}$ & $K_{p_2}$ & $K_{p_3}$   \\ \hline
				pólo 1  &        &        &        \\ \hline
				pólo 2  &        &        &        \\ \hline
				pólo 3  &        &        &        \\ \hline
				zero    &        &        &        \\ \hline					  		  
			\end{tabular}
			\endgroup
		\end{center}
		
	    \item Compare a posição dos pólos de malha fechada em relação à posição
              dos pólos de malha aberta.

              \espacoquatro
              		
        \item Esboce abaixo a resposta transitória para uma entrada do tipo degrau unitário.
		\begin{center}
			\begin{tikzpicture}
			\draw [thick,step=0.5, gray!20] (0,0) grid (15,6);
			\end{tikzpicture}
		\end{center}
		
		\item Anote na tabela abaixo os valores obtidos para o tempo de subida $t_r$, tempo de acomodação
		$t_s$, Máximo sobresinal $M_p$ da variável $\omega_o(t)$ e $\max v_p(t)$ (Valor máximo da saída do controlador
		$v_p(t)$)
		\begin{center}
			\begingroup \large
			\setlength{\tabcolsep}{36pt} % Default value: 6pt
			\renewcommand{\arraystretch}{2} % Default value: 1
			\begin{tabular}{|c|c|c|c|c|}\hline
				$T_i$ & $t_r$ &  $t_s$ & $M_p$ & $\max v_p(t)$ \\ \hline
				$T_{i_1}$ & & & & \\ \hline
				$T_{i_2}$ & & & & \\ \hline
				$T_{i_3}$ & & & & \\ \hline		  		  
			\end{tabular}
			\endgroup
		\end{center}
	
	  \item Como as grandezas $t_r$, $t_s$, $M_p$ e $\max v_p(t)$ variam com a variação de $T_i$ ?

       \espacoquatro	
			
	\end{enumerate}
\end{cexercicio}

\section{Controlador proporcional derivativo}

O controlador PD pode ser escrito da seguinte forma:
\begin{equation}
H_1(s) = K_p(1+T_d\frac{s}{\frac{T_d}{N}s+1}),
\end{equation}

O filtro de 1a. ordem que é definido como:
\begin{equation}
\frac{1}{\frac{T_d}{N}s+1},
\end{equation}
é necessário para garantir que a parte derivativa seja um sistema causal (grau do polinômio do denominador
igual ou maior que o grau do polinômio do numerador), o que garante a estabilidade numérica da simulação.
Além disso, a componente derivativa gera valores de saída extremamente elevados para entradas rápidas, por exemplo, do tipo degrau (Para uma entrada degrau a derivada é infinita) ou ruídos de alta frequência.
Quanto menor o valor de $N$ menor a frequência de corte do filtro.

O controlador PD não possui obviamente integrador dessa forma para o sistema de controle de
velocidade apesar de que uma resposta estável e rápida possa ser obtida o erro
erro estático $\omega_{e_{ss}}$ não pode ser feito nulo.

\subsection{Controle de velocidade}
\begin{cexercicio}{}
	\begin{enumerate}
		\item Considerando que:
		$H_\omega(s)=K_{tac}K_p(1+T_d\frac{s}{\frac{T_d}{N}s+1})$ e que
		\begin{equation}
		G_{\omega}(s) = \frac{K_1 K_m K_g}{(s+a)(s+a_m)},
		\end{equation}
		calcule o erro estático $\omega_{e_{ss}}$ do sistema de controle em malha fechada para uma entrada
		degrau $\Omega_i(s)=A/s$.
		
		\espacocinco
		
		\item Utilizando o script de controle de velocidade realize simulações do sistema de 
		controle em malha fechada com três valores do tempo derivativo do controlador
		$T_d=\{ 0.1, 0.5, 1.5 \}$, os outros parâmetros devem ser:
		\begin{itemize}
			\item $K_p = 1.0$,
			\item $N = 1.0$
		\end{itemize}
		
		\item Escreva a função de transferência do sistema em malha aberta com valores numéricos:
		\begin{center}
			\begingroup \large
			\renewcommand{\arraystretch}{4} % Default value: 1
			\begin{tabular}{|c|p{10cm}|}\hline
				\( \displaystyle GH_\omega(s) \)  &   \\ \hline		  		  
			\end{tabular}
			\endgroup
		\end{center}
		
		\item Identifique os pólos de malha aberta $GH_\omega(s)$ para cada um dos valores de $T_d$:
		\begin{center}
			\begingroup \large
			\setlength{\tabcolsep}{30pt} % Default value: 6pt
			\renewcommand{\arraystretch}{2} % Default value: 1
			\begin{tabular}{|c|c|c|c|}\hline
				& $T_{d_1}$ & $T_{d_2}$ & $T_{d_3}$   \\ \hline
				pólo 1  &        &        &        \\ \hline
				pólo 2  &        &        &        \\ \hline
				pólo 3  &        &        &        \\ \hline
				% pólo 4  &        &        &        \\ \hline				
				zero    &        &        &        \\ \hline					  		  
			\end{tabular}
			\endgroup
		\end{center}
		
		\item Escreva a função de transferência em malha fechada do sistema com valores numéricos:
		\begin{center}
			\begingroup \large
			\renewcommand{\arraystretch}{4} % Default value: 1
			\begin{tabular}{|c|p{10cm}|}\hline
				\( \displaystyle \frac{GH_\omega(s)}{1+GH_\omega(s)} \)  &   \\ \hline		  		  
			\end{tabular}
			\endgroup
		\end{center}
		
		\item Identifique os pólos da malha fechada para cada um dos valores de $T_d$:
		\begin{center}
			\begingroup \large
			\setlength{\tabcolsep}{30pt} % Default value: 6pt
			\renewcommand{\arraystretch}{2} % Default value: 1
			\begin{tabular}{|c|c|c|c|}\hline
				& $T_{d_1}$ & $T_{d_2}$ & $T_{d_3}$   \\ \hline
				pólo 1  &        &        &        \\ \hline
				pólo 2  &        &        &        \\ \hline
				pólo 3  &        &        &        \\ \hline
				%pólo 4  &        &        &        \\ \hline				
				zero    &        &        &        \\ \hline					  		  
			\end{tabular}
			\endgroup
		\end{center}
	
	    \item Compare a posição dos pólos de malha fechada em relação à posição
	          dos pólos de malha aberta.
	          
	          \espacoquatro
		
		\item Esboce abaixo a resposta transitória para uma entrada do tipo degrau unitário.
		\begin{center}
			\begin{tikzpicture}
			\draw [thick,step=0.5, gray!20] (0,0) grid (15,6);
			\end{tikzpicture}
		\end{center}
		
		\item Anote na tabela abaixo os valores obtidos para o tempo de subida $t_r$, tempo de acomodação
		$t_s$, Máximo sobresinal $M_p$ da variável $\omega_o(t)$ e $\max v_p(t)$ (Valor máximo da saída do controlador
		$v_p(t)$)
		\begin{center}
			\begingroup \large
			\setlength{\tabcolsep}{36pt} % Default value: 6pt
			\renewcommand{\arraystretch}{2} % Default value: 1
			\begin{tabular}{|c|c|c|c|c|}\hline
				$T_d$ & $t_r$ &  $t_s$ & $M_p$ & $\max v_p(t)$ \\ \hline
				$T_{d_1}$ & & & & \\ \hline
				$T_{d_2}$ & & & & \\ \hline
				$T_{d_3}$ & & & & \\ \hline		  		  
			\end{tabular}
			\endgroup
		\end{center}
		
	  \item Como as grandezas $t_r$, $t_s$, $M_p$ e $\max v_p(t)$ variam com a variação de $T_d$ ?
	  
	  \espacoquatro
	
	\end{enumerate}
\end{cexercicio}

\subsection{Controle de posição}

\begin{cexercicio}{}
	\begin{enumerate}
         \item Considerando que:
         $H_\theta(s)=K_{tac}K_p(1+T_d\frac{s}{\frac{T_d}{N}s+1})$ e que
         \begin{equation}
         G_{\theta}(s) = \frac{K_1 K_m K_g}{s(s+a)(s+a_m)},
         \end{equation}
		calcule o erro estático $\theta_{e_{ss}}$ do sistema de controle em malha fechada para uma entrada
		degrau $\Theta_i(s)=A/s$.
		
		\espacoquatro

		\item Utilizando o script de controle de posição realize simulações do sistema de 
controle em malha fechada com três valores do tempo derivativo do controlador
$T_d=\{ 0.1, 0.5, 1.0\}$, os outros parâmetros devem ser:
\begin{itemize}
	\item $K_p = 10$,
	\item $N = 1$
\end{itemize}		
 		
		\item Escreva a função de transferência em malha aberta do sistema com valores numéricos:
		\begin{center}
			\begingroup \large
			\renewcommand{\arraystretch}{4} % Default value: 1
			\begin{tabular}{|c|p{10cm}|}\hline
				\( \displaystyle GH_\theta(s) \)  &   \\ \hline		  		  
			\end{tabular}
			\endgroup
		\end{center}
		
		\item Identifique os pólos de malha aberta $GH_\theta(s)$ para cada um dos valores de $T_d$:
		\begin{center}
			\begingroup \large
			\setlength{\tabcolsep}{30pt} % Default value: 6pt
			\renewcommand{\arraystretch}{2} % Default value: 1
			\begin{tabular}{|c|c|c|c|}\hline
				$K$  & $T_{d_1}$ & $T_{d_2}$ & $T_{d_3}$   \\ \hline
				pólo 1  &        &        &        \\ \hline
				pólo 2  &        &        &        \\ \hline
				pólo 3  &        &        &        \\ \hline
				pólo 4  &        &        &        \\ \hline
				zero    &        &        &        \\ \hline					  		  
			\end{tabular}
			\endgroup
		\end{center}
		
		\item Escreva a função de transferência em malha fechada do sistema com valores numéricos:
		\begin{center}
			\begingroup \large
			\renewcommand{\arraystretch}{4} % Default value: 1
			\begin{tabular}{|c|p{10cm}|}\hline
				\( \displaystyle \frac{GH_\theta(s)}{1+GH_\theta(s)} \)  &   \\ \hline		  		  
			\end{tabular}
			\endgroup
		\end{center}
		
		\item Identifique os pólos da malha fechada para cada um dos valores de $K_p$:
		\begin{center}
			\begingroup \large
			\setlength{\tabcolsep}{30pt} % Default value: 6pt
			\renewcommand{\arraystretch}{2} % Default value: 1
			\begin{tabular}{|c|c|c|c|}\hline
				$K$  & $T_{d_1}$ & $T_{d_2}$ & $T_{d_3}$   \\ \hline
				pólo 1  &        &        &        \\ \hline
				pólo 2  &        &        &        \\ \hline
				pólo 3  &        &        &        \\ \hline
				pólo 4  &        &        &        \\ \hline				
				zero    &        &        &        \\ \hline					  		  
			\end{tabular}
			\endgroup
		\end{center}
		
	    \item Compare a posição dos pólos de malha fechada em relação à posição
              dos pólos de malha aberta.

              \espacoquatro

		\item Esboce abaixo a resposta transitória para uma entrada do tipo degrau unitário.
		\begin{center}
			\begin{tikzpicture}
			\draw [thick,step=0.5, gray!20] (0,0) grid (15,6);
			\end{tikzpicture}
		\end{center}
		
		\item Anote na tabela abaixo os valores obtidos para o tempo de subida $t_r$, tempo de acomodação
		$t_s$, Máximo sobresinal $M_p$ da variável $\omega_o(t)$ e $\max v_p(t)$ (Valor máximo da saída do controlador
		$v_p(t)$)
		\begin{center}
			\begingroup \large
			\setlength{\tabcolsep}{36pt} % Default value: 6pt
			\renewcommand{\arraystretch}{2} % Default value: 1
			\begin{tabular}{|c|c|c|c|c|}\hline
				$T_d$ & $t_r$ &  $t_s$ & $M_p$ & $\max v_p(t)$ \\ \hline
				$T_{d_1}$ & & & & \\ \hline
				$T_{d_2}$ & & & & \\ \hline
				$T_{d_3}$ & & & & \\ \hline		  		  
			\end{tabular}
			\endgroup
		\end{center}
	
	\item Como as grandezas $t_r$, $t_s$, $M_p$ e $\max v_p(t)$ variam com a variação de $T_d$ ?
	\espacoquatro
		
	\end{enumerate}
\end{cexercicio}

\section{Discussões}

\espacocinco\espacodois

\section{Conclusões}

\espacocinco\espacodois

\end{document}
